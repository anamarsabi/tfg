% !TeX root = ../tfg.tex
% !TeX encoding = utf8

\chapter{Appendix: Auxiliary theorems}\label{ap:appendix}

\begin{teorema}[Spectral theorem] \label{th:spectral theorem}
    Let $\mathbb{H}$ be a separable Hilbert space and let T be a compact self-adjoint operator. Then there exists a Hilbert basis ( a orthonormal basis) composed of eigenvectors of T. \cite{brezis2011functional} Theorem 6.11
\end{teorema}

\begin{teorema}[Hausdorff theorem] \label{th:Hausdorff theorem}
    Every linear biyective application between finite dimension normed spaces is an isomorphism. \cite{paya-apuntes}
\end{teorema}

\begin{teorema}[Riesz-Fréchet representation theorem] \label{th:Riesz theorem}
    Let $\mathbb{H}$ be a Hilbert space and $\Lambda \in \mathbb{H}^{*}$. Then it exists a unique $y \in \mathbb{H}$ such that $\Lambda = \Lambda_y$, that is, $\Lambda(x) = \braket{x,y}\quad \forall x \in \mathbb{H}$.

    Moreover, the application 
    \begin{align}
        \mathbb{H} &\longrightarrow \mathbb{H}^{*}\\
        y &\longmapsto \Lambda_y
    \end{align}
    is an isometry (isomorphism that preserves the norm: $||\Lambda|| = ||y||$ ) and it is linear conjugate. \cite{brezis2011functional} Theorem 5.5
\end{teorema} 

\begin{teorema}[Orthogonal projection theorem] \label{th:orthogonal projection theorem}
    Given $\mathbb{H}$ a Hilbert space and $\mathcal{Y}$ a closed subset in $\mathbb{H}$. Then,
    \begin{enumerate}
        \item The orthogonal projection $P_{\mathcal{Y}}$ from $\mathbb{H}$ to $\mathcal{Y}$ verifies that $\forall x \in \mathbb{H}$, $P(x)$ is the only point in $\mathcal{Y}$ so that $|| P(x) - x|| = \mathrm{dist}(x,\mathcal{Y})$. Therefore, $P(x) = P_{\mathcal{Y}}(x)$ 
        \item $\forall x \in \mathbb{H}$, $||x||^2 = ||P_{\mathcal{Y}}(x)||^2 + ||x - P_{\mathcal{Y}}(x)||^2$. In particular, $P_{\mathcal{Y}}$ is continuous.
        \item $\mathcal{X} = \mathcal{Y} \oplus \mathcal{Y}^{\perp}$
        \item $\mathcal{Y}^{\perp \perp} = \mathcal{Y}$ and $P_{\mathcal{Y}^{\perp}} = \mathrm{id}_{\mathcal{H}} - P_{\mathcal{Y}}$
    \end{enumerate}
    \cite{mariamedina-apuntes}
\end{teorema}

\begin{teorema}[Schmidt decomposition]\label{th:schmidt decomposition}
    Suppose $\ket{\psi}$ is a pure state of a composite sytem, AB. Then there exist orthonormal states $\ket{i_A}$ for system $A$, and orthornormal states $\ket{i_B}$ of system $B$ such that 
    \begin{equation}
        \ket{\psi} = \sum_{i} \lambda i_i \ket{i_A} \ket{i_B}
    \end{equation}

    where $\lambda_i$ are non-negative real numbers satisfying $\sum_i \lambda_i^2=1$.
    
    \cite{nielsen_chuang_2010} Theorem 2.7
\end{teorema}



\endinput
%------------------------------------------------------------------------------------
% FIN DEL APÉNDICE. 
%------------------------------------------------------------------------------------
% !TeX root = ../tfg.tex
% !TeX encoding = utf8

\chapter{Conclusion}\label{ch:conclusion}
In this thesis, we have explored from a mathematical point of view supervised machine learning principles, focusing on support vector machines and neural networks. We have presented the quantum mechanics model and formulated the theoretical framework of quantum machine learning, targeting quantum support vector machines and quantum neural networks with whom we have also worked in the practical part. The experimentation and implementation section of this thesis was fruitfully completed.  We believe that we've largely achieved our primary goals as set out initially.

Throughout chapters \ref{ch:1-MathematicalFundamentalsML} and \ref{ch:2-IntroQC} we have acquired a solid understanding of the two main components of the puzzle: supervised machine learning and quantum computing. We have followed the objectives we had set for each chapter. This has allowed us to later focus on quantum machine learning on chapter $\ref{ch:3-QML}$. Understanding this field from CQ perspective (classical data and quantum computation) we have studied quantum support vector machines and quantum neural networks mathematical framework, completing the objectives for this chapter. This effort towards understanding supervised machine learning, quantum computing and its synergies in quantum machine learning was rewarded for the elaboration of chapter's \ref{ch:4-Implementation} practical work. 

Chapter \ref{ch:4-Implementation} has been a really stimulating chapter. Not only it served its purpose of getting acquainted with PennyLane and the implementation of the quantum machine learning models studied in theory, but also the quantum nature of the problem and the results obtained were remarkable. We understood the non-trivial problem of detecting quantum entanglement and its subtleties through looking at it from a quantum machine learning lens. As Richard Feynman suggested at his opening speech ``Simulating Physics with computers'' \cite{Feynman}, by looking at the problem from a quantum computing perspective, we gained some valuable insights about the problem itself. Additionally, we achieved up to $96.9\%$ of accuracy with quantum support vector machines and $80.6\%$ of accuracy with our best configuration of quantum neural network proved to be satisfactory results for our first interaction with a quantum machine learning application on a real world scenario. This proved that not only the theory of quantum machine learning is developed, but also the tools and resources are ready to be used for their implementation at least at a small scale due to the constraints by simulators and limited number of qubits.

Regarding future areas of development of this work, we present several approaches. Firstly, this work could be continued shifting the focus to unsupervised machine learning and its translation to quantum computing to complete the panorama of machine learning and its quantum counterpart. 

Secondly, the case study on detection of quantum entanglement has room for further research on several ways. Quantum neural networks have more aspects to consider (its architecture and variational form) and it can be performed a more exhaustive analysis, especially if working with a device with GPU acceleration and more resources to deal with the long training times. The availability of more powerful simulators and larger quantum systems (with more qubits) will also open new possibilities for the difficulties we had in trying data embedding techniques, reducing the dimensionality of the dataset and training our models. It would also add a supplementary dimension to the problem to run the quantum machine learning models on real quantum hardware such as IBM's quantum computers.

Lastly, PennyLane's great interoperability with TensorFlow unlocks the possibility of building hybrid architectures: models that combine classical models with other quantum-based models by joining them together and training them as a single unit. In particular, we have hybrid quantum neural networks at one's fingertips thanks to Keras which is a valuable tool for their implementation since they are classical neural networks in which one or more of its layers have been replaced by quantum layers. 

To conclude, it is deserved to carry on with the research activity in quantum machine learning and we encourage to take part in it, taking advantage of current frameworks such as PennyLane based on this thesis promising results. With research and the evolution of quantum resources, we will build and find out the role of this discipline, keeping in mind that there is no fight for the future between classical and quantum machine learning. Quoting Nelson Mandela, ``I never lose, I either win or learn''


\endinput
%--------------------------------------------------------------------
% FIN DEL CAPÍTULO. 
%--------------------------------------------------------------------
